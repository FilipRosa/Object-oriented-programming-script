\section{Třídy a objekty}

\lstset{
  language=C++,                % Nastavení jazyka pro zvýraznění syntaxe
  basicstyle=\ttfamily\small,  % Nastavení písma pro kód
  keywordstyle=\color{blue},    % Barva pro klíčová slova
  commentstyle=\color{green},  % Barva pro komentáře
  stringstyle=\color{red},     % Barva pro řetězce
  numbers=left,                % Čísla řádků na levé straně
  numberstyle=\tiny\color{gray},
  stepnumber=1,                % Čísla řádků se budou zobrazovat každý řádek
  numbersep=5pt,               % Vzdálenost mezi čísly řádků a kódem
  frame=single,                % Rám kolem kódu
  tabsize=4,                   % Velikost tabulátoru
  showspaces=false,            % Nezobrazovat mezery
  showstringspaces=false,      % Nezobrazovat mezery v řetězcích
  breaklines=true,             % Zalamování dlouhých řádků
}

\subsection{Co je hlavními příčinami potřeby změn software?}
\begin{enumerate}
    \item \textbf{Technologický pokrok}
    \item \textbf{Zmeny v užívateľkých požiadavkách}
    \item \textbf{Bezpečnosť}
    \item \textbf{Chyby a problémy}
    \item \textbf{Zmeny v legislatíve}
    \item \textbf{Zastaralosť softwaru}
    \item \textbf{Zlepšenie užívateľského zážitku}
    \item \textbf{Prechod na cloud alebo inú architektúru}
    \item \textbf{Konkurenčný tlak}
\end{enumerate}

\subsection{Jaké jsou hlavní faktory ovlivňující objektovou orientovanost?}
\begin{itemize}
    \item metóda a jazyk
    \item implementácia a prostredie
    \item knižnice
\end{itemize}


\subsection{Vysvětlete, co rozumíme pojmy objektově orientovaná metoda (přístup) a jazyk.}
Nejde len o programovací jazyk a spôsob jeho použitia, ide aj o spôsob uvažovania a vyjadrovania a taktiež o záznamy v textovej alebo grafickej forme.

\begin{itemize}
    \item trieda
    \item trieda ako modul
    \item trieda ako typ
    \item zasielanie správ
    \item skrývanie informácií
    \item statická kontrola typov
    \item dedičnosť, redefinícia, polymorfizmus, dynamická väzba
    \item generickosť
    \item správa pamäti a garbage collector
\end{itemize}

\subsection{Vysvětlete, co rozumíme podporou objektově orientované implementace.}
Je to podpora vývoja. Zahŕňa vlastnosti a efektivitu nástrojov pre vývoj, nástroje pre podporu nasadenia nových verzií a nástroje pre podporu dokumentovania.


\subsection{Vysvětlete, co rozumíme podporou opakované použitelnosti.}
Znemená to navrhovanie a vytváranie komponént, modulov, kódu alebo systému takým spôsobom, aby boli ľahko použiteľné v rôznych kontextoch alebo projektoch bez nutnosti ich opätovného vytvárania alebo prepisovania.

Príklady:
\begin{itemize}
    \item Knižnice a frameworky
    \item Kódové šablóny
    \item Cloudové služby a API
\end{itemize}


\subsection{Vysvětlete pojmy třída a objekt a použijte správnou terminologii.}
\textbf{Trieda} je časť software, ktorá popisuje abstraktný dátový typ a jeho implementáciu. \newline
\textbf{Objekt} je abstraktný dátový typ so spoločným chovaním reprezentovaným zoznamom operácií, ktoré vie objekt vykonávať.


\subsection{Zdůrazněte vlastnosti třídy z pohledu modularity.}
Triedy nepopisujú len typy objektov, ale musia byť zároveň modulárnymi jednotkami. V čisto OOP by nemali byť iné samostatné jednotky než triedy.


\subsection{Vysvětlete princip zapouzdření v OOP.}
Zapúzdrenie (enkapsulácia) obmedzuje prístup k určitým častiam objektu a poskytuje kontrolu nad tým, ako sú dáta a funkcie v objekte využívané. \newline
Cieľom je ochrana dát a zjednodušenie správy komplexivity. Pre interakciu s dátami objektu a jeho ovládanie sa používajú metódy \textbf{Getter} a \textbf{Setter}.


\subsection{Vysvětlete princip zasílání zpráv.}
Tento princíp slúži na komunikáciu medzi objektami aplikácie. Namiesto toho, aby objekty priamo pristupovali k internému stavu iných objektov alebo priamo volali ich metódy, komunikujú medzi sebou prostredníctvom správ. \newline
Správy predstavujú požiadavky na vykonanie nejakej akcie, ktorú objekt vykonaná na základe prijatej správy. Tieto správy môžu byť volania metód, žiadosti o zmenu stavu objektu alebo žiadosť o dáta.


\subsection{Vysvětlete principy deklarace a definice jednoduché třídy v C++.}
\textbf{Deklarácia} je proces, pri ktorom definujeme štruktúru triedy, teda aké vlastnosti a chovanie trieda bude mať.
class MyClass {
public: // Specifikuje, že členy jsou veřejné (přístupné zvenčí)
    MyClass(); // Konstruktor
    void display(); // Veřejná metoda
private: // Specifikuje, že členy jsou soukromé (nepřístupné zvenčí)
    int x; // Soukromý atribut
};
\newline
\textbf{Definícia} je proces, pri ktorom konkrétne určujeme ako budú metódy triedy implementované.
