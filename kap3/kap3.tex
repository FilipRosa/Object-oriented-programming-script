\section{Návrh programu I}

\subsection{Vysvětlete, jak vznikají objekty třídy, pojem konstruktor a principy práce s ním v C++.}
Objekt je inštanciou triedy. Je reprezentáciou nejakej entity, ktorá má stav reprezentovaný dátami a chovanie reprezentované metódami.\newline
Vzniká konštruktorom, ktorý inicializuje objekty. Nemá návratovú hodnotu, pokiaľ nie je deklarovaný, vytvorí sa automaticky. Pri statickej deklarácii sa volá automaticky, inak je nutné použiť \textbf{new}. Môže ich byť viac, ale musia sa líšiť počtom alebo typom parametrov.


\subsection{Vysvětlete, jak zanikají objekty třídy, pojem destruktor a principy práce s ním v C++.}
Objekty zanikajú deštruktorom. Ten slúži pre dealokovanie dynamicky vytvorenej pamäti. Nemá návratovú hodnotu. Pokiaľ nie je definovaný, vytvorí sa automaticky. Pri statickej deklarácii sa volá automaticky, inak s použitím \textbf{delete}.


\subsection{Vysvětlete rozdíl mezi statickou a dynamickou deklarací objektů v C++.}
\textbf{Statická deklarácia:} objekt je na zásobníku, jeho životnosť je automatická a riadená blokom, v ktorom bol vytvorený. \newline
\textbf{Dynamická deklarácia:} objekt je na halde, jeho životnosť je riadená manuálne pomocou new a delete.


\subsection{Jak se dá postupovat, pokud chceme v zadání programu nalézt třídy, jejich metody a datové
členy?}
\begin{itemize}
    \item \textbf{triedy} - často opakujúce sa podstatné mená
    \item \textbf{metódy} - slovesá
    \item \textbf{dáta} - podstatné mená
\end{itemize}


\subsection{Kdy a proč potřebujeme použit více konstruktorů jedné třídy?}
V prípade, že potrebujeme viacero inštancií danej triedy s rôznym počtom alebo typom parametrov.


\subsection{Kdy potřebujeme deklarovat a definovat destruktor?}
Keď sú dáta objektu dynamické.


\subsection{Co jsou výchozí konstruktory a destruktory a k čemu je potřebujeme?}
Východzí konštruktor je automaticky definovaný kompilátorom, pokiaľ nie je explicitne deklarovaný. Slúži k vytvoreniu objektu s východzími hodnotami. \newline
Východzí deštruktor je automaticky definovaný kompilátorom, pokiaľ nie je explicitne deklarovaný. Slúži k zničeniu objektu bez špecifickej činnosti.


\subsection{Jaké typy metod obvykle musíme deklarovat a definovat?}
\begin{itemize}
    \item konštruktor a deštruktor
    \item metódy poskytujúce informácie o stave objektu (getter)
    \item metódy meniace stav objektu (setter)
\end{itemize}


\subsection{Co jsou objektové kompozice a k čemu jsou dobré?}
Objekt môže byť súčasťou iného objektu a stáva sa jeho dátovou položkou. \newline
Vznikajú tak komponované objekty s presne definovanými kompetenciami. Tie však môžu byť realizované prostredníctvom iterácie objektov, z ktorých sú komponované.